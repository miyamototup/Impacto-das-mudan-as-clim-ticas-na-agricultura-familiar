
\documentclass[12pt,a4paper]{elsarticle}

%How to remove the line "Preprint submitted to Elsevier"
%As linhas abaixo servem para remover o comentário de rodapé que vem automaticamente com o template da elsevier
\makeatletter
\def\ps@pprintTitle{%
	\let\@oddhead\@empty
	\let\@evenhead\@empty
	\let\@oddfoot\@empty
	\let\@evenfoot\@oddfoot
}
\makeatother


\usepackage[a4paper,top=2cm, bottom=2cm,right=2cm,left=2cm]{geometry}

\journal{Journal Name}

\begin{document}

	\begin{frontmatter}


		
		\title{Climate change impacts on Brazilian smallholder agriculture: A spatial panel data analysis for maize, bean, cassava and milk }
		
		\author{Bruno Cesar Brito Miyamoto\fnref{1}\corref{cor1}}
   	   %Esse comando adiciona e-mail adress no final da página
        \ead{miyamototup@gmail.com}
		\author{Alexandre Gori Maia\fnref{1}\corref{cor1}}
	   	   %Esse comando adiciona e-mail adress no final da página
	     \ead{gori@unicamp.br}
	   	   
	    %Adress se refere e fnref{1}
	   \address[1]{UNICAMP,Brasil}
	   
		%cortex{cor1} coloca titulo no rodapéo de corresponding author e coloca asterisco nesse título tb 
		%esse asterisco e por causa do comando \corref{cor1}
		\cortext[cor1]{Corresponding author}
		
	
		%Adicionando mais de um autor com especificação de email
		%\author{coco coco\corref{cor1}\fnref{1}} 
		%\ead{coco.coco@gmail.com}
		%\author{momo momo\corref{cor1}\fnref{2}}
		%\ead{momo.momo@gmail.com}
		%\cortext[cor1]{Corresponding author}
		
		%Adicionando apenas uma instituição
		%\address{UNICAMP,BRAZIL}
		
		%Adicionando mais de uma instituição
		%\address[1]{university 1}
		%\address[2]{university 2} 
		

		\begin{abstract}
			
			pest and diseases control, variety pf crop that could be raised in some particular areas, farming practices, extreme whether events : heat waves, droughts, strong winds and heavy rains
		-Climate change - climate change impacts on agriculture - impacts on brazilian agriculture -  how those changes will affect the most vulnerable producers (smallholders famers) -  importance of smallholder for brazilian food production -  objective of the paper - methodology (climatic data interpolation - k means clustering - econometric data panel model)- main results - contribution of the paper
		

		\end{abstract}
		
		\begin{keyword}
			a \sep b \sep c
		\end{keyword}

	\end{frontmatter}



\end{document}



